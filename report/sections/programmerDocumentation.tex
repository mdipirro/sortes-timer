\section{Documentation for programmers}
\subsection{Specification}
The aim of this program is to simulate an alarm clock. The functionality it provides is:
\begin{itemize}
	\item Displaying the time flowing in the following format: \texttt{hh:mm:ss};
	\item When the time reaches the value 23:59:59, it jumps to 00:00:00;
	\item When the board is powered up, the user is asked to set both the clock's time and the wake up time;
	\item The board does not ring when it wakes up. Instead, a led blinks for 30 seconds;
	\item When the clock is running it is possible to change both the clock's and the wake up time without influencing the clock.
\end{itemize}

\subsection{Structural choices}
The program operates by using interruptions. 

\textbf{BUT1} and \textbf{BUT2} are mapped to high priority interruptions, respectively \textbf{INT3} and \textbf{INT1}. The function \texttt{high\_isr} handles a pressure on one of these two buttons. Since it is a bad practice to have ``fat'' interrupt handlers, the function only acts on a flag to let the \texttt{main} know what happened. These two flags are the fields \texttt{button1} and \texttt{button2} of the \texttt{interrupts} struct type. The \texttt{main} then acts accordingly, by calling either \texttt{handle\_button1\_pressure} or \texttt{handle\_button2\_pressure}. Their behavior depends on the actual state of the timer. On one hand they are used to set both the clock's and the wake up time. On the other hand, they are used to begin a setting procedure. The action is determined by one of the following three flags, forming the \texttt{flags} struct type:
\begin{itemize}
	\item \texttt{time\_setting\_procedure}, 1 if and only if the user is setting the clock's time;
	\item \texttt{awake\_setting\_procedure}, 1 if and only if the user is setting the wake up time;
	\item \texttt{set}, \ if and only if the initial setting procedure has been completed.
\end{itemize}

Please notice that once \texttt{flags.set} is set at 1, it never changes this value. On the other hand, both \texttt{time\_setting\_procedure} and \texttt{awake\_setting\_procedure} are reset when the corresponding setting procedure ends, and set at 1 one it restarts.

Three different struct types are used to store time values:
\begin{itemize}
	\item \texttt{clock\_time} represents the flowing time and is shown when no setting procedure is underway. It contains three fields: \texttt{hours}, \texttt{minutes} and \texttt{seconds};
	\item \texttt{awake\_time} represents the wake up time. This struct only contains \texttt{hours} and \texttt{minutes};
	\item \texttt{setting\_values} represents a time value which will be copied into \texttt{clock\_time} at the end of the setting procedure. We use a separate struct for this so that the current, ongoing, time value is not affected by the user. On a different version providing the possibility to cancel a setting procedure, this copy would be useful in order to preserve the ongoing value to be modified. Since the user is not allowed to set the seconds, this struct only contains two fields: \texttt{hours} and \texttt{minutes}.
\end{itemize}

We use a char array, \texttt{time\_value}, to store the time values as a string. This array is then used during the LCD display updates. The main entry point for these updates is the function \texttt{update\_display}. It accepts one parameter representing the current timer's state. This state is one of the three values of the \texttt{display\_states} enum type:
\begin{itemize}
	\item \texttt{CLOCK\_SETTING}, corresponding to the clock's time setting procedure;
	\item \texttt{TIMER\_SETTING}, corresponding to the awake time setting procedure;
	\item \texttt{TIME\_FLOWING}, corresponding to the time-only view;
\end{itemize}
