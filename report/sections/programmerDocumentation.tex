\section{Documentation for programmers}
\subsection{Specification}
The aim of this program is to simulate an alarm clock. The functionality it provides is the following:
\begin{itemize}
	\item Displaying the time flowing in the following format: \texttt{hh:mm:ss};
	\item When the time reaches the value 23:59:59, it jumps to 00:00:00;
	\item When the board is powered up, the user is asked to set both the current time and the wake up time;
	\item The board does not ring when it wakes up. Instead, a led blinks for 30 seconds;
	\item When the clock is running it is possible to change both the current and the wake up time without influencing the clock.
\end{itemize}

\subsection{Structural choices}
The program operates by using interrupts.

\textbf{BUT1} and \textbf{BUT2} are mapped to high priority interrupts, respectively \textbf{INT3} and \textbf{INT1}. The function \texttt{HighISR} handles a pressure on one of these two buttons. Since it is a bad practice to have ``fat'' interrupt handlers, the function only acts on two flags to let the \texttt{main} know what happened. These flags are the fields \texttt{button1} and \texttt{button2} of the \texttt{interrupts} struct type. The \texttt{main} then acts accordingly, by calling either \texttt{HandleButton1Pressure} or \\\texttt{HandleButton2Pressure}. Their behavior depends on the actual state of the timer. On one hand they are used to set both the current and the wake up time. On the other hand, they are used to begin a setting procedure. The action is determined by one of the following three flags, forming the \texttt{flags} struct type:
\begin{itemize}
	\item \texttt{time\_setting\_procedure}, high if and only if the user is setting the current time;
	\item \texttt{awake\_setting\_procedure}, high if and only if the user is setting the wake up time;
	\item \texttt{set}, high if and only if the initial setting procedure has been completed.
\end{itemize}

Please notice that once \texttt{flags.set} is set at 1, it never changes this value. On the other hand, both \texttt{time\_setting\_procedure} and \texttt{awake\_setting\_procedure} are reset when the corresponding setting procedure ends, and set at 1 when it restarts.

Three different struct types are used to store time values:
\begin{itemize}
	\item \texttt{clock\_time} represents the flowing time and is shown when no setting procedure is underway. It contains three fields: \texttt{hours}, \texttt{minutes} and \texttt{seconds};
	\item \texttt{awake\_time} represents the wake up time. This struct only contains \texttt{hours} and \texttt{minutes};
	\item \texttt{setting\_values} represents a time value which will be copied into \\\texttt{clock\_time} at the end of the setting procedure. We use a separate struct for this so that the current, ongoing, time value is not affected by the user. On a different version providing the possibility to cancel a setting procedure, this copy would be useful in order to preserve the ongoing value to be modified. Since the user is not allowed to set the seconds, this struct only contains two fields: \texttt{hours} and \texttt{minutes}.
\end{itemize}

We use a char array, \texttt{time\_value}, to store the time values as a string. This array is then used during the LCD display updates. The main entry point for these updates is the function \texttt{UpdateDisplay}. It accepts one parameter representing the current timer state. This state is one of the three values of the \texttt{display\_states} enum type:
\begin{itemize}
	\item \texttt{CLOCK\_SETTING}, corresponding to the current time setting procedure;
	\item \texttt{TIMER\_SETTING}, corresponding to the awake time setting procedure;
	\item \texttt{TIME\_FLOWING}, corresponding to the time-only view;
\end{itemize}
When needed, we update this array using the function \texttt{ultoa} and then we display the new time. This function is helpful, because it inserts a null character (\texttt{`$\backslash$0'}) after the converted value, such that the display update does not require the entire array to be shown. For instance, when the user is setting the hours value only two digits (and consequently two characters) have to be displayed: there is no need to display the minutes value, which remains unchanged during the hours setting. 

Another possibility would be to use \texttt{sprintf} instead of \texttt{ultoa}. The former simplifies the string formatting, since it is possible to format the value with two digits (by providing \texttt{$``$\%2d$"$} as a parameter). Nonetheless, the use (and consequently the inclusion) of this function increases the space occupied by the program. We achieve the same result with an additional check on the generated string. We enforce by constructions that the values to be converted are always valid time values (range 0-59). Consequently, the generated string has always at least two characters: the number (in case it is less than 10) and the null character. By accessing the second character (which is always well defined), and comparing it with \texttt{`$\backslash$0'} we can discriminate two different cases:
\begin{enumerate}
	\item They are equal. The value comprises only one digit and we need to prepend (in \texttt{time\_value}) a $0$;
	\item They are not equal. The value comprises two digits and there is no need to further manipulate its char representation.
\end{enumerate}
The \texttt{time\_value} dereferentiation correctness is ensured by construction. We initialize and manipulate its elements making sure we never access an uninitialized element or an out-of-bounds location.\\

The time flow is managed using \texttt{timer1}, linked to the external \textit{32.768 kHz} oscillator (see appendix A). This is a commonly used value in digital clocks because it is a power of two ($2^{15}$), allowing to easily count in seconds.

In this software the timer counter generates an interrupt every 0.5 seconds; this means that a 16 bit register is needed and that the counter is not free-running, because the initial value must be set every time an overflow occurs. The time overhead of initial value reassignation, however, is completely negligible.

A different flag is set for half-seconds (\texttt{flags.half\_sec}) and full-seconds (\texttt{flags.one\_sec}), both are handled by the main function.
The program executes a series of instructions when the half-second flag is high: it flips the yellow LED value, then if a second has passed it calls the function \texttt{UpdateClock} to ensure that the global variables and the display reflect the change in time; furthermore, it handles the blinking of the red LED according to the state of the alarm (``ringing'' or not). In the end, the flags are reset to zero.

Some efforts have been put in trying to make the two LEDs blink as synchronously as possible: this was achieved by smart ordering of some instructions and \texttt{if} statements de-annidation.