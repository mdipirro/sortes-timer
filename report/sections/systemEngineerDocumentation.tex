\section{Documentation for system engineers}
\subsection{Compilation and download}
The program comes with a \texttt{Makefile} that can be used to compile the program. To this end, the right command to use is \texttt{make timer}. It will generate some files in the current directory (\texttt{.}) and in \texttt{./Objects/}. In order to download the program into the naked computer, the user should follow the following steps (supposing the router is correctly configured):
\begin{itemize}
	\item Run \texttt{tftp 192.168.97.60} in the same directory as \texttt{timer.c}. The \texttt{tftp} environment will start;
	\item \texttt{binary}, to send the program as binary;
	\item \texttt{trace}, to see what happens;
	\item \texttt{verbose}, to see more information;
	\item \texttt{put timer.hex}.
\end{itemize}
Please notice that the last command has to be run only when the board is ready to receive the program. this happens during the three seconds after its reboot.

\subsection{Testing}